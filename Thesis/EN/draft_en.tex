% THESIS DRAFT (EN)

\documentclass[a4paper,english]{article}
\usepackage[utf8]{inputenc}
\usepackage{graphicx}

\title{
    Open Sensor Networks\\
    Initial Report
}

\author{
    Alejandro Andreu Isábal\\
    \texttt{alejandro.andreu01@estudiant.upf.edu}
}


% BEGIN DOCUMENT
\begin{document}

    \maketitle

    \section{Introduction}

        During the last decade inside the Internet revolution, we the users only got the chance to know one model of how things can be done. That is, a company reserves and manages the resources for the client to enjoy. It is quite obvious than within this approach the end-user remains completely passive.

        This is where \emph{bottom-up broadband} comes in play. It aims to turn the old model upside down. The \emph{OSN} pilot follows as well this approach, where the end user has an active role in the development, deployment and management of this network. Without a central authority, only the beneficiaries would be the ones who conform the system.

        The \emph{bottom-up broadband} (\emph{BuB} from now on) has several advantages. This kind of network is managed and configured easier and faster since it does not have a central point of failure, as the \emph{P2P} model has proven many times. Also, because of the underlying openness and transparency of this approach it can be adapted to specific needs, which leads to community growing.


    \subsection{Objectives}

        The main objective of this pilot is to deploy a sensor network under the \emph{BuB} model. Through the completion of the project its viability will also be studied.

        \begin{itemize}
            \item Gather interesting environmental information.
            \item Make the citizenship aware of the environmental conditions they live in, and also make them want to be a part of it.
            \item Elaborate a network model where each participant adds resources, this be an individual or a public institution, etc.
        \end{itemize}

        In a nutshell, to make citizens realize that knowing our environment can improve their daily lives. It is expected that at the pilot completion a node ---wich would contain several sensors--- will be working \emph{out-of-the-box} thus uploading data automatically to the Internet.


    \subsection{Similar initiatives}

        There are some initiatives that do not follow exactly this approach but they have gained fame and acknowledgement. Some of them are listed below.

        \subsubsection{Libelium -- \texttt{www.libelium.com}}
            
            Libelium is a sensor platform that is not focused just on smart cities but also on agriculture, car detection, events, gas detection, etc. They use a wide variety of protocols to enable communication such as ZigBee, Wi-Fi, Bluetooth, GSM/GPRS, NFC, etc. Their main architecture is a wireless sensor network that uploads data trough an Internet gateway.

            However, they are a commercial and closed solution. They do not intend to upload the information to open data portals but to private databases. Also, their product is not open and cannot be modified to do whatever the end user wants, although they give an API to modify the behavior.

        \subsubsection{Air Quality Egg -- \texttt{www.airqualityegg.com}}

            This company provides an egg-shaped base station which is powered by Ethernet and outdoors sensors that communicate through radiofrequency with "the egg". Real-time data is recollected and then uploaded to \emph{Cosm} (former Pachube).

            They have a good \emph{wiki} with all the sensors they use to measure air quality as well as other documents to help understand the platform. 

            \subsubsection{SmartCitizen -- \texttt{www.smartcitizen.me}}

            SmartCitizen is an initiative that was born in Barcelona some time ago and pursues almost the same objective as this pilot. They succesfully launched a crowdfunding campaign and they are near delivering the first prototypes. Their platform heavily relies on a web application to display the information as well as in \emph{Cosm}. They also believe in openness and transparency.

            Their SCK ---\emph{Smart Citizen Kit}--- relies on Wi-Fi and measures environmental factors such as carbon dioxide, humidity, nitrogen dioxide, etc. They also provide a \emph{social} side of this utility allowing users to share measurements in social networks.

            




% END DOCUMENT
\end{document}
