% Chapter 1

\chapter{Introduction} % Main chapter title

\label{Chapter1} % For referencing the chapter elsewhere, use \ref{Chapter1} 

\lhead{Chapter 1. \emph{Introduction}} % This is for the header on each page - perhaps a shortened title

%----------------------------------------------------------------------------------------

During the last years we, the Internet users, just had one chance to know how things are managed, from top to bottom. That is, telecom operators reserve some resources (optic fiber, certain bandwidth, etc.) for each one of their clients and charge them for this service. In a top-down approach the consumer remains completely passive and has to solemnly accept what the telco dictates. Now, a new model pretends to turn this trend upside down.

This new way to do things is called \emph{bottom-up broadband}, and is also how this project is posed. The very same users that were before passive will become very active, helping not only by designing the network but also deploying and maintaining it, thus participating in every step of the system lifecycle. Hence, without a central authority the usufructuaries are the only ones that conform this kind of networks.

Bottom-up broadband (BuB from now on) schemes have several important advantages over those that follow a conventional top-down approach, such as: easier and faster setup due to the lack of a central authority (as it happens in peer-to-peer networks), it can be adapted to anyone's needs since they are the caretakers of the system, and could also become the solution to those that live in an area that is not economically attractive to regular ISPs\citep{}. % Citar paper Jaume

As for disadvantages, a BuB network creation can be very time consuming, since users participate in every single step of this endeavor\citep{}. % Paper Jaume

Sensor networks are very important nowadays and its objective is to gather data. 
\quote{Data itself isn't good nor bad. Data just represents the surrounding reality. The more data we may access, the more accurate model we may create of the reality, thereby also define our actions in ways that are maximally beneficial to our aims}\citep{}. % PaRaZiTe

But, does it make sense building such a sytem under a BuB model? The answer is yes, it does. As it happened with traditional telecom operators, the information that is obtained through sensor networks is kept by the agencies that own them without even making a public API\footnote{Explicar API?} to ``play'' with this data.

Therefore the main goal of this project is to design and deploy a sensor network that gathers real-time information and that enables developers to create applications that will ultimatelly help the citizenship improve their daily lives. Intrinsically this can be divided in more specific objectives, such as:

\begin{itemize}
    \item Allow citizens, individuals belonging an organization or even enterprises to connect scattered sensor nodes.
    \item Collect different kinds of information and transmit it to the Internet.
    \item Samples (values of sensors) must be gathered ofen enough to be almost real-time.
    \item Use of open technologies to allow easier replication and modification as well as reducing final costs.
    \item The project shall become a tool so anyone that needs or wants to deploy a sensor network can do it as soon as possible.
\end{itemize}

Chapter 2 (\ref{Chapter2}) takes a look at the state of the art. That is, why are sensor network important nowadays and what has been done until now regarding commercial and open solutions.

Chapter 3 (\ref{Chapter3}) shows what technologies have been used to complete this endeavor. A brief description about each element is attached so the reader can replicate more easily the network and have some details at first glance.

Chapter 4 (\ref{Chapter4}) focuses on the way this pilot has been completed. That is, the methodology that has been followed, as well as how problems have been confronted.

In Chapter 5 (\ref{Chapter5}) we can see how is each node and the designed and also how programs and scripts work, mainly through flow diagrams.
