% Chapter 1

\chapter{Introduction} % Main chapter title

\label{Chapter1} % For referencing the chapter elsewhere, use \ref{Chapter1} 

\lhead{Chapter 1. \emph{Introduction}} % This is for the header on each page - perhaps a shortened title

%----------------------------------------------------------------------------------------

\section{Description of the project}

During the last years, we the Internet users just had one chance to know how things are managed, from top to bottom. That is, telecom operators reserve some resources (optic fiber, certain bandwidth, etc.) for each one of their clients and charge them for this service. In a top-down approach the consumer remains completely passive and has to solemnly accept what the telco dictates. Now, a new model pretends to turn this trend upside down.

This new way to do things is called \emph{bottom-up broadband}, and is also how this project is posed. The very same users that were before passive will become very active, helping not only by designing the network but also deploying and maintaining it, thus participating in every step of the system lifecycle. Hence, without a central authority the usufructuaries are the only ones that conform this kind of networks.

Bottom-up broadband (BuB from now on) schemes have several important advantages over those that follow a conventional top-down approach, such as: easier and faster setup due to the lack of a central authority (as it happens in peer-to-peer networks), it can be adapted to anyone's needs since they are the caretakers of the system, and could also become the solution to those that live in an area that is not economically attractive to regular ISPs\citep{}. % Citar paper Jaume

As for disadvantages, a BuB network creation can be very time consuming, since users participate in every single step of this endeavor\citep{}. % Paper Jaume

Sensor networks are very important nowadays and its important to remark that their only objective is to gather data. \emph{Data itself isn't good nor bad. Data just represents the surrounding reality. The more data we may access, the more accurate model we may create of the reality, thereby also define our actions in ways that are maximally beneficial to our aims}\citep{}. % PaRaZiTe

The main goal of this project is to design and deploy a sensor network that gathers real-time information and that enables developers to create applications that will ultimatelly help the citizenship improve their daily lives.

\section{Objectives}

As stated before, the main goal of this project is to enable developers to create apps, websites, etc. That make use of real-time gathered data. Intrinsically this can be divided in more specific objectives, such as:

\begin{itemize}
    \item Allow citizens, individuals belonging an organization or even enterprises to connect scattered sensor nodes.
    \item Collect different kinds of information and transmit it to the Internet.
    \item Samples (values of sensors) must be gathered ofen enough to be almost real-time.
    \item The project shall become a tool so anyone that needs or wants to deploy a sensor network can do it as soon as possible.
\end{itemize}

\section{Specific regulatory implications}

The main problems that can take place in the completion of this project are related to data privacy. This complex topic slightly varies from country to country. Howevery, most mesurable factors can be publicly obtained so it will not pose a problem. For other kinds of information, such a camera with pattern detection software (for instance, to check wether a parking spot is empty or not), data may have to be pre-processed.

%\section{Sustainability and growth}
%These factors are not in the hands of the maintainers of the network but on the developers that use the information that the system gathers. The most representative example could be a public administration that, using various sources of data (involved citizens) developed a mobile application for their citizenship to enjoy and make their daily lives better.

\section{Spectrum regulations}
The project shall be available to any organisation, institution or individual that wants to deploy a sensor network. It is also multi-purpose and when it comes to the hardware it will be using Digi XBee\textregistered{} devices, so the portions of the spectrum that are exploited are ISM bands.%Depending on the desired range different frequencies can be used, such as: 2.4GHz, 900MHz or 868MHz.

%\section{Traditional telecom operators offering similar services}
%This service is not offered by any telecom operator at the moment. Due to the nature of the network these are not interested in deploying one, because although it generates growth it is not the kind they seek.

%\section{Funding scheme}
%The project does not need any initial funding since it is completely built from bottom to top, as explained before. The final users are the interested ones in deploying such a network thus they pay for the necessary equipment as well as for the maintenance.
