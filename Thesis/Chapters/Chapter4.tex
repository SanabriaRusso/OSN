% Chapter Template

\chapter{Methodology} % Main chapter title

\label{Chapter4} % Change X to a consecutive number; for referencing this chapter elsewhere, use \ref{ChapterX}

\lhead{Chapter 4. \emph{Methodology}} % Change X to a consecutive number; this is for the header on each page - perhaps a shortened title

%----------------------------------------------------------------------------------------
%	JUST INTRO
%----------------------------------------------------------------------------------------

The first step I took to complete this project was having a deep look at the state of the art. There are a lot of sensor network designs and some are as well open sourced, but the majority of them require either advanced knowledge on PCB fabrication or are focused on just one particular area (they aim to solve just one problem). Thus developing a system which is multipurpose and uses well-known technologies for rapid deployment are some of the key requirements that this network should meet (apart from the initial objectives).

Once all initial requirements are identified I had to choose one appropriate life cycle for the project. The pilot scope was not strictly constrained thus changes shall be handled in some way. Consequently, I chose an \emph{agile} approach.

Agile management is a special case of iterative management, driven by changes\citep{pmbok_agile}. Development of small modules is the usual thing, with deadlines every two or four weeks. Also, stakeholders are highly involved which is very related to the approach we followed all the components of BuB4EU. Each month there was a scheduled workshop where every participant informed the rest of the team about his last advancements. This method is very useful for getting constant feedback hence improving the overall quality of the project.

Also, when problems emerged we have an available mailing list\footnote{Hosted in Guifi.net servers, accessible by entering \url{https://llistes.guifi.net/sympa/arc/bub}.}. There, each participant can propose whatever he/she wants, but also ask questions and await for answers. This way, we have achieved a solid level of collaboration.

At the same time, the pilot followed an open development model, since it was ---and still is--- available on GitHub from the beginning\footnote{The repository can be found at \url{https://github.com/aandreuisabal/OSN}. To see the latest changes, check out the ``develop'' branch.}. I decided to go with a complete open model because this way I give back something to the community, since I am benefiting so much from it to complete this project.
