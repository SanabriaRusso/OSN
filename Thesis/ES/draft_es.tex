%
% Final thesis draft -- ES
%
% Author:
% Alejandro Andreu Isábal
%

\documentclass[a4paper]{article}
\usepackage[utf8]{inputenc}
\usepackage{comment}

\title{Open Sensor Network}
\author{Alejandro Andreu Isábal}

\begin{document}

\maketitle

\section{Introducción}

En los últimos años dentro del mundo de la banda ancha los usuarios sólo hemos tenido la oportunidad de conocer un enfoque, el de la teleoperadora que reserva y gestiona los recursos para el uso y disfrute del cliente. Es evidente que en este tipo de planteamiento el consumidor toma un papel totalmente pasivo.

Así, el rumbo que persigue el enfoque \emph{bottom-up broadband}, dentro del cual está incluido este piloto, es todo lo contrario de lo explicado anteriormente. El usuario final pasaría a contribuir activamente al desarrollo de este tipo de redes, tanto en el desplegamiento como en el mantenimiento. Sin una autoridad central serían los beneficiarios los que formen en su totalidad este tipo de redes.

Este modelo presenta una serie de ventajas. Por ejemplo, al no tener una autoridad central la red se gestiona y configura más rápida y fácilmente, tal y como sucede en las arquitecturas P2P. Además, la red se puede ajustar a unas necesidades específicas, lo que fomentaría todavía más su implementación.


\subsection{Objetivos}
El objetivo principal de este proyecto es desplegar una red de sensores bajo el modelo \emph{BuB}. A través de este proyecto también se estudiará la viabilidad de este tipo de redes.

\begin{itemize}
    \item Recolectar datos que puedan resultar de interés.
    \item Concienciar a los ciudadanos de las condiciones medioambientales en las que viven, ya que normalmente en las ciudades éstas dejan que desear.
    \item Elaborar un modelo de red en que cada participante aporte recursos, sea un particular o un organismo público.
\end{itemize}

En definitiva, hacer ver que conocer nuestro alrededor puede mejorar nuestro día a día. Se espera que al final del proyecto se haya diseñado un nodo compuesto por diversos sensores, que funcione \emph{out-of-the-box} y que sea capaz de subir automáticamente los datos que recolecte a Internet.


\subsection{Integración dentro de BuB}
El hecho de que esta WSN --red inalámbrica de sensores-- siga este enfoque plantea una cuestión principal, ¿cómo aporta cada parte algo a la red?

Aumentar el número de nodos que componen la red no se traduciría en otra cosa que más resolución a la hora de obtener datos, con la posibilidad que los núcleos urbanos donde se desplegara la red llegaran a convertirse en lo que se llama una \emph{living city}, es decir, una ciudad donde sus elementos "hablan" entre ellos.


\section{Commons for Europe}
Este piloto está enmarcado dentro del proyecto europeo \emph{Commons for Europe}, que a su vez nació gracias al gran éxito de la organización \emph{Code for America}. El objetivo final de C4EU es mejorar las ciudades europeas gracias a las nuevas tecnologías y las oportunidades que éstas traen consigo.\\

Al mismo tiempo, esta organización europea está dividida en dos ramas:
\begin{description}
    \item[CodeCommons] Se centra en desarrollar proyectos web así como aplicaciones móviles de manera colaborativa y transparente.
    \item[BuBCommons] Como se ha explicado anteriormente explora las posibilidades que ofrece el modelo \emph{bottom-up}, utilizando tecnologías como Super Wi-Fi, fibra óptica, Wi-Fi y sensores.
\end{description}

\end{document}
