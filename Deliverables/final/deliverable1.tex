% 1ST DELIVERABLE -- TFG

\documentclass[a4paper,english]{article}
\usepackage[utf8]{inputenc}
\usepackage{graphicx}

\title{
    Open Sensor Networks\\
    Project Planning
}

\author{
    Alejandro Andreu Isábal\\
    \texttt{alejandro.andreu01@estudiant.upf.edu}
}


% BEGIN DOCUMENT
\begin{document}

    \maketitle

    \section{Tasks}

        The following tasks have been identified as critical in order for the pilot to be completed.

        \subsection{Identify a topology}

            The topology determines many of the future decisions that will shape the project. There are many possibilities, each one with its own advantages and disadvantages. Also, the underlying protocols that will alow the messages to be sent can accept a limited number of topologies.

            
        \subsection{Analyze available protocols}

            Once a working topology has been identified there probably are many protocols that meet the requirements. Since this is a sensor network it is clear that the factors that have to be minimized are:

            \begin{description}
                \item[Energy consumption] This is the most critical factor and each node should have some degree of self-sufficiency. The power source should be an end-user decision and that's why this is such an important factor.
                \item[Scalability] Ideally the network would grow over time and it has to be able to accept all the requests as well as handling a bigger number of nodes.
            \end{description}


        \subsection{Platform to implement the network}
            
            As well as in the previous section, there are a big number of platforms nowadays that can accomplish the task. A good start point is looking for OSH ---\emph{open-source hardware}--- because it is fully configurable and cheaper than commercial solutions. Of course, it has to meet the previous requirements in order for this to work.


        \subsection{Gathered data}

            The data that must be recollected have to be identified. This would be the considered the \emph{default} data, because each user has the freedom to add/remove the sensors he wants. Initially this data would be mainly environmental with the idea in mind that this could be adapted to any situation with the adequate knowledge. The corresponding sensors will be then bought to test the measurements and go forward with the project.


        \subsection{Point-to-point communication}

            Approaching the first milestone, there must be a communication between node and sink. It does not involve any other kind of communication but just from node to sink.


        \subsection{Data uploading}

            It is not enough to communicate between nodes but also, as the objective of the pilot states, make data publicly available. Depending on the protocol there will be a need of external scripts or not. In case there is, it will be implemented in \emph{Python} due to its simplicity and flexibility. It would act as a kind of API between open data portals and the sink.


        \subsection{Network scalation}

        Once all this work is completed it's time to add more nodes to the network and maybe even do some simulation with software like \emph{NS2}. Scalability is a very important factor as stated earlier. If necessary, some code readjustments would be necessary.


        \subsection{Web application}

            Although all the data is uploaded to the Internet, it is necessary to write a little web application to directly show where the nodes are located. This gives a better overview of the network status. This of course, would be synchronized with all the open data portals.


        \subsection{Project documentation}

            This is an essential documentation in case someone external to the project wants to replicate the project and/or adapt it to his needs. Since every aspect of this sensor network will be completely open, documentation will be easy to understand and write.


        \subsection{Pilot thesis and presentation}

            Final tasks. Necessary for the pilot to be graded in some way. They also complement the defense.



    \section{Timing}

        The following figure is the Gantt chart of the remaining tasks of the project and how they are scheduled. Right after this figure and to summarize all the previous information, there is the WBS ---work breakdown structure---.

        \begin{center}
            \includegraphics[scale=0.5,angle=90]{pics/gantt1.png}
        \end{center}


        \begin{center}
            \includegraphics[scale=0.5,angle=90]{pics/gantt2.png}
        \end{center}


        \begin{center}
            \includegraphics[scale=0.5,angle=90]{pics/gantt3.png}
        \end{center}


        \begin{center}
            \includegraphics[scale=0.5,angle=90]{pics/wbs.png}
        \end{center}
            











% END DOCUMENT
\end{document}
